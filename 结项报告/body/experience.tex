% !Mode:: "TeX:UTF-8"

\chapter{实践心得}

\section*{郑志嘉}
在本次实践中,小组三个人几乎完整地完成了一整个项目的编写,采用前后端分离的开发模式也更加符合当下真实开发的情境,这次实践项目做下来让我收获颇丰,最重要的就是对于实际开发流程有了深刻的了解。除此之外,由于此次开发我担任的是后端开发以及数据库管理,这让我对JDBC有了更好的掌握,也学习到了Servlet和Springboot框架的开发流程。另外,我对于前端的开发也有了一定的了解,对前后端结合的方式也更有把握,这对我以后的开发工作有非常重要的帮助。总而言之,本次实践经历非常有意义,不论是对我的团队协作能力还是开发水平都有显著的提升。~\\

\section*{周圣喻}
在本次综合实践中,我主要负责的是前端部分的编写。我之前几乎没有系统地学习过前端相关的知识,这对我来说是一个非常大的挑战。我在实践过程中跟随着课程讲解边学边做,获得了非常大的收获,现在已经可以独立编写一个网页的前端部分。除此之外,这也是我第一次以团队合作的方式做一个项目。在团队一起完成项目时,小组成员之前相互讨论,相互进步,我也体验到团队合作的方式,了解到如何多人同时进行一个项目。总的来说,我在本次实践中收获巨大,进步许多,代码能力和团队合作能力有了显著提升。~\\

\section*{梁益铭}
在本次实践中,我主要负责后端的开发。由于之前没有接触过前后端分离项目的开发,在一步步学习完成整个项目后非常有成就感,对于软件开发有了一定程度的理解,为以后的学习工作打下了坚实的基础。由浅到深,由JDBC项目到Servlet项目,最后再到SpringBoot项目,一点点的深入学习使我对整个开发框架有了十分详细的了解。在后续新功能的开发中,进一步将所学的知识加以实践,与前端队友合作完成接口的制定,进行前后端分离的开发过程,最终成功将自己所想的功能一一实现,受益匪浅。