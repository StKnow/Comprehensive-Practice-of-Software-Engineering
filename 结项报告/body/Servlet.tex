% !Mode:: "TeX:UTF-8"

\chapter{Servlet 项目}

\section{项目需求}

本项目要求使用VUE+Servlet+AJAX技术,开发前后端分离的饿了么Web应用程序。主要设计参照 “饿了么官网网页版”制作,并且仅关注点餐业务线功能,“饿了么官网”中的其它功能暂不涉及。

本项目需要完成的功能大致分为十一个页面:

\subsection{首页功能}
\subsubsection*{\normalsize主要功能}
显示点餐分类信息
\subsubsection*{\normalsize动作}
1.点击点餐分类小图片,跳转到商家列表页面

2.点击下方菜单栏中的“订单”,跳转到历史订单页面

\subsection{商家列表页面功能}
\subsubsection*{\normalsize主要功能}
1.根据点餐分类显示商家列表信息

2.如果处于登录状态,那些需要查询购物车中是否有此商家的食品。如果有,在页面上显示食品数量
\subsubsection*{\normalsize动作}
点击某个商家,跳转到此商家的详细信息页面

\subsection{商家详细信息页面功能}
\subsubsection*{\normalsize主要功能}
显示商家详细信息及所属食品信息,并自动计算总价
\subsubsection*{\normalsize动作}
1.点击某食品的 + 按钮,食品数量加 1

2.点击某食品的 - 按钮,食品数量减 1

3.点击“去结算”按钮,跳转到确认订单页面

4.如果没有登录,那么上述三个动作会自动跳转到登录页面

\subsection{确认订单页面功能}
\subsubsection*{\normalsize主要功能}
1.确认订单信息是否正确

2.选择送货地址
\subsubsection*{\normalsize动作}
1.点击送货地址,跳转到送货地址列表页面

2.点击“去支付”按钮,跳转到支付页面

\subsection{在线支付页面功能}
\subsubsection*{\normalsize主要功能}
显示订单信息及订单明细信息
\subsubsection*{\normalsize动作}
无(点餐业务流程到此结束)

\subsection{送货地址列表页面功能}
\subsubsection*{\normalsize主要功能}
显示当前用户的送货地址信息
\subsubsection*{\normalsize动作}
1.点击某个送货地址,跳转回确认订单页面

2.点击“新增”按钮,跳转到新增送货地址页面

3.点击“编辑”按钮,跳转到送货地址编辑页面

4.点击“删除”按钮,删除此送货地址

\subsection{新增送货地址页面功能}
\subsubsection*{\normalsize主要功能}
添加新的送货地址
\subsubsection*{\normalsize动作}
点击“保存”按钮,添加新的送货地址,并跳转到送货地址列表页面

\subsection{编辑送货地址页面功能}
\subsubsection*{\normalsize主要功能}
编辑送货地址
\subsubsection*{\normalsize动作}
点击“更新”按钮,保存编辑后的送货地址,并跳转到送货地址列表页面

\subsection{登录页面功能}
\subsubsection*{\normalsize主要功能}
用户登录
\subsubsection*{\normalsize动作}
1、 点击“登陆”按钮,进行登陆业务处理。如果登陆成功,跳转到上一个页面

2、 点击“去注册”按钮,跳转到注册页面

\subsection{注册页面功能}
\subsubsection*{\normalsize主要功能}
注册新用户
\subsubsection*{\normalsize动作}
点击“注册”按钮,进行新用户注册。注册成功后,跳转到登陆页面

\subsection{历史订单页面功能}
\subsubsection*{\normalsize主要功能}
显示用户历史订单信息
\subsubsection*{\normalsize动作}
点击某个历史订单,可以对订单明细信息进行显示和隐藏

\begin{figure}[H]
    \centering
    \subfigure{
        \begin{minipage}[t]{0.22\linewidth}
            \centering
            \includegraphics[width=3.2cm,height=6.4cm]{figures/3.1.1.png}\\
            \includegraphics[width=3.2cm,height=6.4cm]{figures/3.1.2.png}\\
            \includegraphics[width=3.2cm,height=6.4cm]{figures/3.1.9.png}\\
        \end{minipage}
    }
    \subfigure{
        \begin{minipage}[t]{0.22\linewidth}
            \centering
            \includegraphics[width=3.2cm,height=6.4cm]{figures/3.1.3.png}\\
            \includegraphics[width=3.2cm,height=6.4cm]{figures/3.1.4.png}\\
            \includegraphics[width=3.2cm,height=6.4cm]{figures/3.1.10.png}\\
        \end{minipage}
    }
    \subfigure{
        \begin{minipage}[t]{0.22\linewidth}
            \centering
            \includegraphics[width=3.2cm,height=6.4cm]{figures/3.1.5.png}\\
            \includegraphics[width=3.2cm,height=6.4cm]{figures/3.1.6.png}\\
            \includegraphics[width=3.2cm,height=6.4cm]{figures/3.1.11.png}\\
        \end{minipage}
    }
    \subfigure{
        \begin{minipage}[t]{0.22\linewidth}
            \centering
            \includegraphics[width=3.2cm,height=6.4cm]{figures/3.1.7.png}\\
            \includegraphics[width=3.2cm,height=6.4cm]{figures/3.1.8.png}\\
        \end{minipage}
    }
    \centering
    \caption{VUE前端页面实现结果}
\end{figure}

\section{项目设计}

\subsection{VUE前端设计}
首先安装cnpm和VueCli开发环境,使用VSCode工具进行项目开发。

接着搭建VueCli模板工程、添加依赖及配置文件。添加font-awesome与axios依赖。
添加图片到src的assets中。
在src目录下添加common.js文件,用于添加一些公用方法。
在工程根目录下添加vue.config.js文件,用于配置前端的端口号。
添加main.js文件,用于判断每个页面是否需要登录才可以使用。
添加App.vue文件,里面设置全局的共通样式。

最后实现各个组件:主页、商家列表、商家详细信息、订单、支付、添加修改用户地址、历史订单、登录、注册和底部导航栏共同组件。

\subsection{数据库设计}
本项目使用MySQL数据库,选择DataGrip作为开发工具。在数据表的设计上,需要创建商家表、食品表、购物车表、送货地址表、订单表、订单明细表、用户表共七张表。
\begin{figure}[H]
    \centering
    \includegraphics[width=15cm,height=17cm]{figures/table2.jpg}
    \caption{Servlet项目数据表}
\end{figure}

\subsection{服务器端设计}



